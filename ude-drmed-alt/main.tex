\documentclass{article}
\usepackage[utf8]{inputenc}
\usepackage[german]{babel}
\usepackage[T1]{fontenc}
% \usepackage[backend=biber, style=ude-authoryear-addon, maxnames=9999, maxcitenames=2]{biblatex}
\usepackage[backend=biber, style=ude-authoryear-addon]{biblatex}
\addbibresource{literatur.bib}


\begin{document}

\section{Examples as they should look like}

\subsection{@article}

\begin{quote}
Martin, E., Russell, D., Goodwin, S., Chapman, R., North, M., Sheridan, P. (1991):\\
Why patients consult and what happens when they do.\\
Br. Med. J. \underline{303}, 289-292.
\end{quote}

\subsection{@book 1}

Books with not more than three authors.

\begin{quote}
Spiro, H.M. (1977):\\
Clinical Gastroenterology. 2. Ed.\\
New York: Macmillan.
\end{quote}

\begin{quote}
Burck, K.B., Liu, E.T., Larrick, J.W. (1988):\\
Oncogenes: An Introduction to the Concept of Cancer Genes.\\
New York, Berlin, Heidelberg (usw.):\\
Springer-Verl.; s. bes. S. 99.
\end{quote}

\subsection{@book 2}

\begin{quote}
Goust, M.M. (1990): Major histocompatibility complex.\\
\underline{In:} Virella, G., Goust, J.M., Fudenberg, H.H. (Eds.): Introduction to Medical Immunology.\\
2. Ed. Immunology Ser. Vol. 50; S. 31-51. New York, Basel: Dekker
\end{quote}

\section{What I did so far}

My first approach was to use extend standard.bbx by creating an own bbx-file. The results are quite ok and I put them here: \url=https://github.com/firefish475/wiki-rli/tree/main/ude-drmed-alt=

A lot of stuff shouldn't be really good, as I focussed on re-building the required format without paying any attention to additional information which could be included in the bibliography. So my next approach was to use a biblatex-style which more closely resembles what is required and try to tweak it. This biblatex-style would be authoryear.bbx. Unfortunately so far not even changing the ``.'' by a colon works so far.


\section{Criteria the biblatex style in development should meet}

\section{This section is in german}

Hier ist ein Beispiel für ein Zitat von Zeitschriftenaufsätzen vom entrytype article \parencite{martin1991}.

Außerdem zitieren wir noch Bücher, und zwar Verfasserwerke wie \cite{spiro1977} und \cite{burck1988}
und Herausgeberwerke wie \cite{goust1990} vom entrytype incollection.

The incollection entrytype isn't printed as expected.

\begin{quote}
Goust, M.M. (1990): Major histocompatibility complex.\\
In: Virella, G., Goust, J.M., Fudenberg, H.H. (Eds.): Introduction to Medical Immunology.\\
2. Ed. Immunology Ser. Vol. 50; S. 31-51. New York, Basel: Dekker
\end{quote}


\subsection{Angaben zum Aussehen von extern}

\begin{quote}
Das alphabetische Literaturverzeichnis muss entsprechend der
Zitierweise des Index Medicus erstellt und durchnumeriert wer-
den.
\end{quote}

\begin{quote}
Zitieren im Text: Namen von 1 - 2 Verfassern oder bei mehr als 2 Verfassern Na-
me des Erstautors et al., danach Jahreszahl der Publikation.
\end{quote}

\begin{quote}
	1. Zeitschriftenaufsätze
Autorennamen; Initialen nachgestellt; Erscheinungsjahr in runden Klammern; Doppelpunkt; voll-
ständiger Aufsatztitel (bei englischsprachigen Aufsatztiteln Kleinschreibung); Zeitschriftentitel ab-
gekürzt lt. "Index Medicus"; Band bzw. Jahrgang (unterstrichen); Komma; erste und letzte Seite des
Artikels.
Beispiel: Martin, E., Russell, D., Goodwin, S., Chapman, R.,
North, M., Sheridan, P. (1991):
Why patients consult and what happens when they do.
Br. Med. J. 303, 289-292.
2. Bücher (Verfasserwerke von höchstens drei Verfassern)
Verfassernamen; Initialen nachgestellt; Erscheinungsjahr in runden Klammern; Doppelpunkt; Buch-
titel (auch bei enlischsprachigen Buchtiteln Großschreibung); Auflage (ab 2. Aufl.); Erschei-
nungsort (bei mehr als drei nach dem ersten Erscheinungsort "(usw.)" ergänzen); Doppelpunkt;
Verlag; Semikolon; evtl. Seitenhinweis.
Beispiele: Spiro, H.M. (1977):
Clinical Gastroenterology. 2. Ed.
New York: Macmillan.
Burck, K.B., Liu, E.T., Larrick, J.W. (1988):
Oncogenes: An Introduction to the Concept of Cancer
Genes. New York, Berlin, Heidelberg (usw.):
Springer-Verl.; s. bes. S. 99.Anlage 2b
3. Buchbeiträge (Herausgeberwerke)
Alle Autorennamen; Erscheinungsjahr in runden Klammern; Doppelpunkt; Aufsatztitel; Punkt; da-
nach "In:"; sämtliche Herausgebernamen; in runden Klammern "Hrsg." bzw. "Ed." oder "Eds." an-
fügen; Doppelpunkt; Buchtitel; evtl. Reihentitel mit Bandangabe; erste und letzte Seite des zitierten
Artikels (Anders als bei den Zeitschriftenzitaten wird der Seitenangabe ein"S." vorangestellt), Ver-
lagsort; Doppelpunkt; Verlag.
Beispiel: Goust, M.M. (1990): Major histocompatibility complex.
In: Virella, G., Goust, J.M., Fudenberg, H.H. (Eds.): Introduction to Medical Im-
munology.
2. Ed. Immunology Ser. Vol. 50; S. 31-51. New York,
Basel: Dekker
4. Online Zitate
im Text wie bei anderen Titeln auch den Autor erwähnen:
'Miller et al. 2003' und dann im Literaturverzeichnis die online - Quelle
im Detail auflisten. Das heisst, dort muss stehen :
Miller G, Schulz C, Meier F: Die Therapie des Tumorleidens. Onkologie 2003,
Online-Publikation; www.Onkologie/ausgabe123/3
Der Leser muss in der Lage sein, die zitierten Arbeiten/Angaben ohne weitere Recherchen im Inter-
net zu finden. Die Autoren werden im Text wie bei konventionellen Zitaten genannt (Miller et al.
2003)
\end{quote}

\printbibliography

\end{document}

